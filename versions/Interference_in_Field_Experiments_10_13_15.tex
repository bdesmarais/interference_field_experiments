\documentclass[12pt]{article}

%==============Packages & Commands==============
\usepackage{graphicx}
\usepackage{fancyvrb}
\usepackage{tikz}
%%%<
\usepackage{verbatim}
%\usepackage[active,tightpage]{preview}
%\PreviewEnvironment{tikzpicture}
%\setlength\PreviewBorder{5pt}%

\usepackage{geometry}                		% See geometry.pdf to learn the layout options. There are lots.
\geometry{letterpaper}                   		% ... or a4paper or a5paper or ...
%\geometry{landscape}                		% Activat\usetikzlibrary{arrows}e for for rotated page geometry
%\usepackage[parfill]{parskip}    		% Activate to begin paragraphs with an empty line rather than an indent
\usepackage{graphicx}				% Use pdf, png, jpg, or eps§ with pdflatex; use eps in DVI mode
								% TeX will automatically convert eps --> pdf in pdflatex
\usepackage{amssymb}

\usepackage[ruled,vlined]{algorithm2e}
\usetikzlibrary{arrows}
\usepackage{alltt}
\usepackage[T1]{fontenc}
\usepackage[utf8]{inputenc}
\usepackage{indentfirst}
\usepackage[longnamesfirst]{natbib} % For references
\bibpunct{(}{)}{;}{a}{}{,} % Reference punctuation
\usepackage{changepage}
\usepackage{setspace}
\usepackage{booktabs} % For tables
\usepackage{rotating} % For sideways tables/figures
\usepackage{amsmath}
\usepackage{multirow}
\usepackage{color}
\usepackage{dcolumn}
\usepackage{comment}
%\usepackage{fullwidth}
\newcolumntype{d}[1]{D{.}{\cdot}{#1}}
\newcolumntype{.}{D{.}{.}{-1}}
\newcolumntype{3}{D{.}{.}{3}}
\newcolumntype{4}{D{.}{.}{4}}
\newcolumntype{5}{D{.}{.}{5}}
\usepackage{float}
\usepackage[hyphens]{url}
%\usepackage[margin = 1.25in]{geometry}
%\usepackage[nolists,figuresfirst]{endfloat} % Figures and tables at the end
\usepackage{subfig}
\captionsetup[subfloat]{position = top, font = normalsize} % For sub-figure captions
\usepackage{fancyhdr}
%\makeatletter
%\def\url@leostyle{%
%  \@ifundefined{selectfont}{\def\UrlFont{\sf}}{\def\UrlFont{\small\ttfamily}}}
%\makeatother
%% Now actually use the newly defined style.
\urlstyle{same}
\usepackage{times}
\usepackage{mathptmx}
%\usepackage[colorlinks = true,
%						bookmarksopen = true,
%						pagebackref = true,
%						linkcolor = black,
%						citecolor = black,
% 					urlcolor = black]{hyperref}
%\usepackage[all]{hypcap}
%\urlstyle{same}
\newcommand{\fnote}[1]{\footnote{\normalsize{#1}}} % 12 pt, double spaced footnotes
\def\citeapos#1{\citeauthor{#1}'s (\citeyear{#1})}
\def\citeaposs#1{\citeauthor{#1}' (\citeyear{#1})}
\newcommand{\bm}[1]{\boldsymbol{#1}} %makes bold math symbols easier
\newcommand{\R}{\textsf{R}\space} %R in textsf font
\newcommand{\netinf}{\texttt{NetInf}\space} %R in textsf font
\newcommand{\iid}{i.i.d} %shorthand for iid
\newcommand{\cites}{{\bf \textcolor{red}{CITES}}} %shorthand for iid
%\usepackage[compact]{titlesec}
%\titlespacing{\section}{0pt}{*0}{*0}
%\titlespacing{\subsection}{0pt}{*0}{*0}
%\titlespacing{\subsubsection}{0pt}{*0}{*0}
%\setlength{\parskip}{0pt}
%\setlength{\parsep}{0pt}
%\setlength{\bibsep}{2pt}
%\renewcommand{\headrulewidth}{0pt}

%\renewcommand{\figureplace}{ % This places [Insert Table X here] and [Insert Figure Y here] in the text
%\begin{center}
%[Insert \figurename~\thepostfig\ here]
%\end{center}}
%\renewcommand{\tableplace}{%
%\begin{center}
%[Insert \tablename~\theposttbl\ here]
%\end{center}}

\newcommand\independent{\protect\mathpalette{\protect\independenT}{\perp}}
\def\independenT#1#2{\mathrel{\rlap{$#1#2$}\mkern2mu{#1#2}}}
\newcommand{\N}{\mathcal{N}}
\newcommand{\Y}{\bm{\mathcal{Y}}}
\newcommand{\bZ}{\bm{Z}}

\usepackage[colorlinks = TRUE, urlcolor = black, linkcolor = black, citecolor = black, pdfstartview = FitV]{hyperref}


%============Article Title, Authors==================
\title{\vspace{-2cm} Considering Interference in Field Experiments } 


\author{ Sayali Phadke \and Bruce Desmarais} \date{\today}



%===================Startup=======================
\begin{document}
\maketitle



%=============Abstract & Keywords==================

\begin{abstract}

 \noindent  We plan to write a paper about how interference hypotheses can and should be used to analyze the results of field experiments in which there is a high probability that subjects interacted between the administration of treatment and the observation of outcomes. \\~\\

\end{abstract}
\thispagestyle{empty}
\doublespacing
% Description of the possible challenges
\section{Introduction}

Networks are integral parts of human interaction and hence social science research. If one unit in a network gets treated, the effect may trickle down throughout network. The currently established framework for causal inference relies on SUTVA (Stable Unit Treatment Value Assumprion). It assumes that whether or not one person/unit/node is treated, does not affect any other unit. However, SUTVA breaks down in a network setting. It is therefore imperative to take the interference structure into account. Rather, in policy planning or designing marketing campaigns, a researcher may be interested in studying the propogation of treatment effect itself.

In field experiments on social groups, interference may be substantial. In this project we intend to study intererence models for randomized experiments coducted on social networks and causal inference basis this.

\subsection{tasks}
\begin{itemize}
\item Points about why it is interesting to study propagation. (BD)
\item Outline of the paper (SP)
\end{itemize}

\section{Background}

Review of relevant methodological work and substance.

\subsection{tasks}
\begin{itemize}
\item Paragraph on each category of papers that serve as relevant background (SP)
\begin{itemize}
\item Interference models (diffusion, propagation) (BD -- find papers)
\item  Experiments on networks (applications) (BD -- find papers)
\item Approaches to inference or estimation with propagation 
\item Potential outcomes framework (SP -- find papers)
\item Review of political networks (legislative networks)
\item Review of field experiments (BD -- find papers)
\end{itemize}
\end{itemize}


\section{Research Design}

We plan to re-analzye data from past field experimental studies to understand how conclusions regarding direct effects and interference effects depend upon 

\subsection{tasks}
\begin{itemize}
\item Develop a list of alternative propagation models to evaluate. (SP)
\end{itemize}


\section{Analysis}

\begin{itemize}

\item Present original results from studies that we replicate.
\item Present results when we assume some form of interference
\item Explore how alternative assumptions regarding interference change results

\end{itemize}

\subsection{tasks}
\begin{itemize}
\item Replicate Nickerson. (SP)
\item Plot the ideological networks. (BD)
\end{itemize}



%===================References======================

\bibliographystyle{apsr}
\bibliography{cause_and_networks}


\end{document}
